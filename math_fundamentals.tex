%!TEX program = xelatex
%!TEX encoding = UTF-8 Unicode  

\documentclass[12pt]{article}
\usepackage{geometry}
\geometry{letterpaper}

\usepackage{fontspec,xltxtra,xunicode}
\usepackage{color}
\defaultfontfeatures{Mapping=tex-text}
\setromanfont{SimSun} %设置中文字体
\XeTeXlinebreaklocale “zh”
\XeTeXlinebreakskip = 0pt plus 1pt minus 0.1pt %文章内中文自动换行


\newfontfamily{\H}{SimHei}
\newfontfamily{\E}{Arial}  %设定新的字体快捷命令


\usepackage{amsmath}
\usepackage{amsfonts}
\usepackage{amssymb}
\usepackage{graphicx}
\title{\H 三道试题}
\author{章政兴}

%\chapter{机器学习的数学基础}
\begin{document}
\emph{机器学习的数学基础 之 三道试题 } By \textbf{章政兴 Zhengxing.Zhang.cn AT gmail.com}\\
1. Gradients and Hessians\\
Recall that a matrix $A\in{R}^{n\times n}$ is symmetric if $A^T=A$, that is, ${A}_{ij}={A}_{ji}$ for all i, j. Also
recall the gradient $\triangledown f(x)$ of a function $f:{R}^{n} \rightarrow n$ which is the n-vector of partial derivatives

$\triangledown f(x) = 
\begin{bmatrix}
\frac { \partial  }{ \partial { x }_{ 1 } }f(x)\\
\vdots\\
\frac { \partial  }{ \partial { x }_{ n }}f(x)
\end{bmatrix}
\quad where \quad x = 
\begin{bmatrix}
{x}_{1} \\ 
\vdots \\
{x}_{n}
\end{bmatrix}
$.\\
The hessian ${\triangledown}^{2}f(x)$ of a function $f:{R}^{n} \rightarrow R$ is the $n \times n$ symmetric matrix of twice partial
derivatives,
$\begin{bmatrix}
\frac{{\partial}^{2}}{\partial {x}_{1}^{2}}f(x) & \frac{{\partial}^{2}}{\partial {x}_{1} {x}_{2}}f(x) & \cdots & \frac{{\partial}^{2}}{\partial {x}_{1} \partial {x}_{n}}f(x) \\
\frac{{\partial}^{2}}{\partial {x}_{2}{x}_{1}}f(x) & \frac{{\partial}^{2}}{\partial {x}_{2}^{2}}f(x) & \cdots & \frac{{\partial}^{2}}{\partial {x}_{2} \partial {x}_{n}}f(x) \\
\vdots & \vdots & \ddots &\vdots \\
\frac{{\partial}^{2}}{\partial {x}_{n}{x}_{1}}f(x) &
\frac{{\partial}^{2}}{\partial {x}_{n}{x}_{2}}f(x) &
\cdots &
\frac{{\partial}^{2}}{\partial {x}_{n}^{2}}f(x)  
\end{bmatrix}$
\\
(a) Let $f(x)=\frac{1}{2}{x}^{T}Ax + {b}^{T}x$ where A is a symmetric matrix and $b \in {R}^{n}$ is a vector. What is $\triangledown f(x)$?\\
\textcolor[rgb]{1,0,0} {
解:$
%==============
A = \begin{bmatrix}
{R}_{1} \\
{R}_{2} \\
\cdots \\
{R}_{n}
\end{bmatrix}
其中 {R}_{i} = \begin{bmatrix}
	{A}_{i1} \quad {A}_{i2} \quad \cdots \quad {A}_{in}
\end{bmatrix} \\
\\
%========================================
设 {f}_{1}(x)
%=========================================
 = {x}^{T}Ax 
%=========================================
 ={x}^{T}(Ax) \quad \cdots 矩阵乘法结合律\\
%===============================
 ={x}^{T}(\begin{bmatrix}
{R}_{1} \\
{R}_{2} \\
\cdots \\
{R}_{n}
 \end{bmatrix}x) 
%=================================
 = {x}^{T}\begin{bmatrix}
 {R}_{1}x \\
 {R}_{2}x \\
 \cdots \\
 {R}_{n}x
 \end{bmatrix} 
%==================================
 = {x}^{T}\begin{bmatrix}
 {R}_{1}x \\
 {R}_{2}x \\
 \cdots \\
 {R}_{n}x
 \end{bmatrix} \cdots {R}_{i}为行向量;x为列向量  \\
%==================================================
 =\begin{bmatrix}
 {x}_{1} & {x}_{2} &  \cdots  & {x}_{n}  \end{bmatrix}
 \begin{bmatrix}
 {R}_{1}x \\ {R}_{2}x \\ \vdots \\ {R}_{n}x  \end{bmatrix}  \\
 %=======
 =\sum_{i=1}^{n}{{x}_{i}{R}_{i}x}
 %=======
 =\sum_{i=1}^{n}{({R}_{i}x){x}_{i}}
 %===
 =\sum_{i=1}^{n}{\sum_{j=1}^{n}{{A}_{ij}{x}_{j}}{x}_{i}}
 %=====
=\sum_{i=1}^{n}{\sum_{j=1}^{n}{{A}_{ij}{x}_{i}}{x}_{j}}
 \\
 \\
所以:
\triangledown {f}_{1}(x)
=\begin{bmatrix}
 \frac{\partial}{\partial {x}_{1}}({f}_{1}(x)) \\
 \frac{\partial}{\partial {x}_{2}}({f}_{1}(x)) \\
 \vdots \\
  \frac{\partial}{\partial {x}_{k}}({f}_{1}(x)) \\
  \vdots \\
   \frac{\partial}{\partial {x}_{n}}({f}_{1}(x)) \\
\end{bmatrix} \\
通项:
\frac{\partial}{\partial {x}_{k}}({f}_{1}(x))
=\frac{\partial}{\partial {x}_{k}}(\sum_{i=1}^{n}{\sum_{j=1}^{n}{{A}_{ij}{x}_{i}}{x}_{j}}) = part1 + part2 + part3 + part4\\
其中:\\
i = k, j = k时为part1;\\
i \neq k, j \neq k时为 part2; \\
i = k, j \neq k时为 part3; \\
i \neq k, j = k时 为 part4。 \\
可得:\\
part1 = \frac{\partial {A}_{kk}{x}_{k}^{2}}{\partial {x}_{k}} 
= 2{A}_{kk}{x}_{k};\\
part2 = 0 \\
part3 = \frac{\partial \sum_{j \neq k}^{n} {A}_{kj} {x}_{k}{x}_{j}}{{x}_{k}} 
= \sum_{j \neq k}^{n}{A}_{kj}{x}_{j} \\
part4 = \frac{\partial \sum_{i \neq k}^{n} {A}_{ik} {x}_{i}{x}_{k}}{{x}_{k}} 
= \sum_{i \neq k}^{n}{A}_{ik}{x}_{i}
= \sum_{j \neq k}^{n}{A}_{jk}{x}_{j}
= \sum_{j \neq k}^{n}{A}_{kj}{x}_{j}  (\because A \quad is \quad symmetric , \therefore {A}_{jk} = {A}_{kj})\\
可得 part4 = part3 \\
\therefore \frac{\partial}{\partial {x}_{k}}({f}_{1}(x)) 
= part1 + 2part3
= 2{A}_{kk}{x}_{k} + 2\sum_{j \neq k}^{n}{A}_{kj}{x}_{j}
= 2\sum_{j = 1}^{n}{A}_{kj}{x}_{j}
= 2\begin{bmatrix}
{R}_{1}{x}_{1} \\
{R}_{2}{x}_{2} \\
\vdots  \\
{R}_{n}{x}_{n} 
\end{bmatrix} \\
=  2\begin{bmatrix}
{R}_{1} \\
{R}_{2} \\
\vdots  \\
{R}_{n} 
\end{bmatrix}x
= 2Ax \\
综上:\frac{\partial}{\partial {x}_{k}}({f}_{1}(x)) = 2Ax \\
\\
设:{f}_{2}(x) 
= {b}^Tx
= \sum_{i=1}^{n}{{b}_{i}{x}_{i}}  \\
则{\tiny }: \triangledown{f}_{2}(x) 
= \frac{\partial}{\partial x} {f}_{2}(x)
= \begin{bmatrix}
\frac{\partial \sum_{i=1}^{n}{{b}_{i}{x}_{i}} }{\partial {x}_{1}} \\
\frac{\partial \sum_{i=1}^{n}{{b}_{i}{x}_{i}} }{\partial {x}_{2}} \\
\vdots \\
\frac{\partial \sum_{i=1}^{n}{{b}_{i}{x}_{i}} }{\partial {x}_{n}} \\
\end{bmatrix}
= \begin{bmatrix}{b}_{1} \\ {b}_{2} \\ \vdots \\ {b}_{n}  \end{bmatrix} = b\\
\\
结论:\triangledown f(x) = \triangledown {f}_{1}(x) + \triangledown {f}_{2}(x) = 2Ax + b \\
\\ 
政兴评论:这是个很形式化的结论,如果x是标量,我们可轻易得到结论:\\
\triangledown (A{x}^{2}+bx) = 2Ax + b \\
由本题推算过程可知:这个结论当x为向量时,依然成立,只是 A{x}^{2}需表述成:{x}^{T}Ax,且A必须\\对称矩阵;bx需表述成{b}^{T}x。
$ 
}
\end{document}
